\documentclass{article}
\usepackage[T1]{fontenc}
\usepackage[utf8]{inputenc}
\usepackage{amsmath}
\usepackage{amsfonts}
\usepackage{amsthm}
\usepackage{amssymb}
\usepackage{stackengine}
\usepackage{minted}
\usepackage{mathtools}
\usepackage{graphicx,wrapfig,tikz}
\usepackage{physics}
\setlength\intextsep{0pt}
\usetikzlibrary{calc,hobby}
\usepackage{pgfplots}
\usepackage{xcolor}
\usepackage{xargs}
\usepackage{kotex}
\usepackage{mathrsfs} %fourier transform
\usetikzlibrary{external}
\usetikzlibrary{arrows}
\usepackage{setspace}
\usetikzlibrary{arrows}
\usepackage{tabstackengine}
\usepackage{enumitem}
\usepackage{longtable}
\usepackage{listings}
\setlist[enumerate]{itemsep=0mm}
\setlist[enumerate]{label={\arabic*.}}

\definecolor{beta-gray}{rgb}{0.9,0.9,0.9}

\newcommand{\lt}{\;\,}
\newcommand{\ft}{\quad\!}
\newcommand{\ot}{\quad\;}
\newcommand{\dt}{\;\,\;\,}
\newcommand{\plm}[1]{\setcounter{enumi}{#1 - 1}}
\newcommand{\nd}{\quad \text{and} \quad}
\newcommand{\inline}[1]{%
    \colorbox{beta-gray}{\lstinline[language=C++]{#1}}%
}

\newenvironment{code}[5]{
    {
        \usemintedstyle{xcode}
        \begin{spacing}{1.0}
            \let\itshape\relax
            \inputminted[frame=single, firstline=#3, lastline=#4]{#1}{#2}
        \end{spacing}
        \vspace{#5}
    }

}

\usepackage[margin=1in]{geometry}

\begin{document}
    \author{권민재, 20190084}
    \title{CSED353 Computer Network Assignment Report}
    \date{May 13, 2020}
    \maketitle

    \setstretch{1.5}

    \section{프로그램 개요}
     이 프로그램은 Python Scoket을 이용하여 SMTP 프로토콜으로 이메일을 보내는 프로그램이다.

    \section{프로그램의 구조 및 알고리즘}
    \subsection{함수}
    \subsubsection{get\_ip()}
    HELO와 EHLO를 보낼 때, 클라이언트의 IP 주소를 보내기 위해 내 IP 주소를 구하는 함수이다.
    기본적으로는 gethostbyname(gethostname())을 이용하지만, 그 반환값이 로컬 호스트인 경우에는 Google Public DNS에 연결하여 내 IP를 알아낸다.
    \subsubsection{show(t, s)}
    현재 보내거나 받은 데이터를 주어진 예시 코드와 같이 raw string으로 출력시켜주는 함수이다.
    t는 데이터의 type, s는 출력할 데이터이다.
    \subsubsection{recv\_print(\_sock, size=1024)}
    socket \_sock에서 size만큼 데이터를 받아서, 받은 데이터를 show를 이용해 출력하는 함수이다.
    \subsubsection{send\_print(\_sock, msg)}
    show를 이용해 보낼 데이터를 출력하고, socket \_sock으로 데이터를 보내는 함수이다.

    \subsection{알고리즘}
    \subsubsection{Establish a TCP connection}
    Assignment 1에서 했던 것 처럼, AF\_INET과 SOCK\_STREAM을 이용하여 socket을 세팅하고 주어진 주소와 포트에 연결한다.

    \subsubsection{Dialogue with the mail server}
    HELO를 이용하여 메일 서버와 최초 통신을 한다. 내 IP를 보내기 위해서 함수 get\_ip()를 이용한다.

    \subsubsection{Login using SMTP authentication}
    STARTTLS를 보내서 TLS 암호화된 통신을 할 것을 알려주고, ssl.wrap\_socket을 이용하여 기존의 소켓이 ssl 통신을 할 수 있도록 만든다.
    그 후, Extended SMTP를 이용하기 위해 ehlo를 보내고, AUTH LOGIN에 username과 password를 base64 인코딩하여 보냄으로써 인증을 한다.

    \subsubsection{Send a e-mail to your POSTECH mailbox}
    mail FROM과 rcpt TO를 이용하여 보내는 사람과 받는 사람을 조회한 후, 메일의 내용을 Subject, From, To를 포함해서 작성한 후 제일 뒤에 \r\n.\r\n을 붙여서 보냄으로써 메일을 발송한다.
    \subsubsection{Destroy the TCP connection}
    quit 신호를 보냄으로써 SMTP 통신을 끝낼 것임을 알리고, wrap\_sock과 sock을 close하여 소켓을 닫는다.

    \section{토론 및 개선}
    \begin{itemize}
        \item 내 IP를 구할 때, gethostbyname(gethostname())를 사용하면 127.0.0.1이 오는 경우가 있는데, 이를 다른 서버에 연결하여 내 IP를 그 소켓에서 얻는 방법으로 해결하였다.
        \item fstring을 이용하여 가독성을 높였다.
    \end{itemize}

\end{document}